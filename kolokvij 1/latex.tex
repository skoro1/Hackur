\documentclass[a4paper, 11pt]{article}
\usepackage[slovene]{babel}
\usepackage[utf8]{inputenc}
\usepackage[T1]{fontenc}
\usepackage{lmodern}
\usepackage{amsfonts}
\usepackage{amsmath}
\usepackage{multirow}
\usepackage[left=4cm, right=4cm, top=3cm, bottom=3cm]{geometry}
\newcommand{\R}[1]{\texttt{#1}}
\title{Praktična predloga za praktičen izpit} 
\author{Fakulteta za matematiko in fiziko Univerze v Ljubljani}
\date{\today}
\begin{document}
\maketitle
% ===================================================================
\begin{abstract}
   V tem delu si bomo pogledali, kako uporabljamo predlogo za generiranje izpitov,
   ki smo jo kvazi-sestavili na Fakulteti za matematiko in fiziko Univerze v Ljubljani.
\end{abstract}
   

% ===================================================================
\tableofcontents

\section{Pisanje preprostih enačb}

Pisanje matematičnih enačb v {\LaTeX}-u zna biti pogosto mučno, še posebej takrat,
ko želimo v eno vrstico napisati več enačb, med sabo ločenih s presledki in tekstom.
V ta namen smo definirali ukaz $\backslash$\R{oureq}, ki sprejme $n$ parametrov,
sestavi pa enačbo v primerni obliki. Na primer
\begin{verbatim}
   \oureq{x^3 + 1 = 0}{in}{x \neq 1}
\end{verbatim}
   
nam sestavi enačbo $(1)$
\begin{equation}
   x^3 + 1 = 0 \quad in \quad x \neq 1
\end{equation}
V kolikor ne želimo imeti oštevilčene enačbe, uporabimo ukaz $\backslash$\R{oureq*}
z istimi parametri. Tako dobimo s primernimi parametri $-$ bralcu prepuščamo,
da ugotovi katerih $-$ enačbo
\[ \int_{-\infty}^{\infty} f(x)dx > 0 \quad \Rightarrow \quad \exists x \in \mathbb{R} : f(x) > 0      \]

% ===================================================================

\section{Matrike in tabele}

Prav tako je zelo nemarno pisanje matrik in tabel. V naslednjih dveh razdelkih
si bomo pogledali, kako nam naša predloga omogoča lažje pisanje le-teh.

\subsection{Pisanje matrik}

Za pisanje matrik, imamo definiran ukaz $\backslash$\R{drawMatrix},
ki sprejme več parametrov, in sicer
\begin{itemize}
   \item  Prvi parameter je \emph{znak}, ki določa robove matrik, recimo [ ali $\{$.
   \item  Naslednja parametra sta število vrstic in stolpcev, recimo jima $n$ in $m$.
   \item  Sledi seznam dolžine $n\cdot m$, v katerem navedemo vsebino celic, po vrsticah.
\end{itemize}
  
   

\textbf{Pomembno:} {\footnotesize UKAZ MORAMO UPORABITI V MATEMATIČNEM NAČINU!}

Tako lahko na zelo preprost način sestavimo naslednjo matriko, s katero imamo drugače kar nekaj težav.
     \begin{center}
      
         
      
      \begin{tabular}{c c c c c}

      1 2 3 ?? m 
      2 4 6 ?? 2m 
      3 6 9 ?? 3m 
      \dots \dots ?? ?? ?? 
      n  2n 3n ?? nm 
      \end{tabular}
       


     \end{center}
      

\subsection{Pisanje tabel}

Pisanje tabel poteka podobno kot pisanje matrik, le da za to uporabimo ukaz $\backslash$\R{drawTable}.
Ta ukaz pa po vrsti sprejme naslednje parametre:
\begin{enumerate}
   \item število vrstic in stolpcev, ločenih z vejico,
   \item niz, ki določa poravnavo posameznih stolpcev in morebitne vmesne črte,
   \item morebitni opis tabele,
   \item morebitna oznaka tabele za sklicevanje,
   \item seznam, v katerem po vrsticah navedemo vsebino celic.
\end{enumerate}
   
   
Z omenjenim ukazom lahko tako zgeneriramo tabelo $1$.
\begin{center}
   
   \begin{tabular}{c|c c}
      
      &cena  &[EUR]\\  \hline
      pivo &1,07 \\
      kruh &0,49 \\
      ocena &$\infty$ \\
  
   \end{tabular}
\end{center}
      
   Primer tabele, ki jo lahko preprosto zgeneriramo z našo predlogo.

% ===================================================================

\end{document}