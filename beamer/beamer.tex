\documentclass{beamer}
\usepackage[utf8]{inputenc}
\usepackage[T1]{fontenc}
\usepackage[slovene]{babel}
\usepackage{lmodern}  
\usepackage{hyperref}
\usetheme{Berlin}
\usecolortheme{default}
\useinnertheme[Shadows]{rounded}
\title{Priprava prosojnic v {\LaTeX}u}
\subtitle{Uporaba paketa beamer}
\author{Matjaž Zaveršnik}
\institute{FMF Fakulteta za matematiko in fiziko}
\begin{document}
% ===================================================================
\frame{\titlepage}


% -------------------------------------------------------------------


% ===================================================================
\begin{frame}
   \frametitle{Kratek pregled}
      \tableofcontents%[pausesections]
       \section {Razporeditev vsebine}   
\end{frame}
% -------------------------------------------------------------------
\begin{frame}
   \frametitle{Naštevanje}
   Za naštevanje lahko uporabimo okolje \emph{itemize}:
      \begin{itemize}
         \item Prva točka.
         \item Druga točka.
         \item Tretja točka.
      \end{itemize}
   ali pa okolje \emph{enumerate}:
   \begin{enumerate}
      \item Prva točka.
      \item Druga točka.
      \item Tretja točka.
   \end{enumerate}
   \end{frame}
% -------------------------------------------------------------------

   Bloki z naslovom
   Dele besedila lahko zapišemo v bloke.
   Uporabimo okolja block, exampleblock, alertblock.
   Za parameter okolja napišemo naslov bloka.
   Opomba
      Tako je videti block z naslovom.
   Primer
      Tako je videti exampleblock z naslovom.
   Opozorilo
      Tako je videti alertblock z naslovom.

% -------------------------------------------------------------------

   Bloki brez naslova
   Blok lahko ima tudi prazen naslov.
   V takem primeru bo brez naslovne vrstice.
      Tako je videti block s praznim naslovom.
      Tako je videti exampleblock s praznim naslovom.
      Tako je videti alertblock s praznim naslovom.

% -------------------------------------------------------------------

   Stolpci
            Besedilo lahko pišemo v več stolpcih.
            Osnovno okolje je columns.
            Posamezen stolpec opišemo v okolju column.
            Vsebina stolpca je lahko poljubna.
            Za primer imamo v desnem stolpcu napis v bloku in sliko sončnice.
            Slika v stolpcu.

% ===================================================================

\section {Matematične trditve}

% -------------------------------------------------------------------

   Praštevila
      Praštevilo je naravno število, ki ima natanko dva delitelja.
   Zgledi
         1 je praštevilo (ima samo enega delitelja: 1).
         2 je praštevilo (ima dva delitelja: 1 in 2).
         3 je praštevilo (ima dva delitelja: 1 in 3).
         4 ni praštevilo (ima tri delitelje: 1, 2 in 4).

% -------------------------------------------------------------------

   Praštevila
      Praštevil je neskončno mnogo.
      Denimo, da je praštevil končno mnogo.
         Naj bo p največje praštevilo.
         Naj bo q produkt števil 1, 2, ??, p.
         Število q+1 ni deljivo z nobenim praštevilom, torej je q+1 praštevilo.
         To je protislovje, saj je q+1>p.

% ===================================================================

\section {Postopno odkrivanje vsebine}

% -------------------------------------------------------------------

   Konstrukcija pravokotnice na premico p skozi točko T
            Dani sta premica p in točka T.
            Nariši lok k s središčem v T.
            Premico p seče v točkah A in B.
            Nariši lok m s središčem v A.
            Nariši lok n s središčem v B in z enakim polmerom.
            Loka se sečeta v točki C.
            Premica skozi točki T in C je pravokotna na p.

% -------------------------------------------------------------------

   Odkrivanje tabele po vrsticah
      Oznaka A B C D
      X 1 2 3 4
      Y 3 4 5 6
      Z 5 6 7 8

% -------------------------------------------------------------------

   Odkrivanje tabele po stolpcih
      Oznaka A B C D
      X 1 2 3 4
      Y 3 4 5 6
      Z 5 6 7 8

% ===================================================================

\section{Razno}

% -------------------------------------------------------------------

% ===================================================================

\end{document}