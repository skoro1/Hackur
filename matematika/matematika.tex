\documentclass[a4paper,12pt]{article}
\usepackage[slovene]{babel}
\usepackage[utf8]{inputenc}
\usepackage[T1]{fontenc}
\usepackage{amsmath}
\usepackage{lmodern}
\usepackage{amssymb}
\usepackage{mathtools}
\newcommand{\bfrac}[2]{\genfrac{(}{)}{0pt}{}{#1}{#2}}
\newcommand{\brfrac}[2]{\genfrac{}{}{0pt}{}{#1}{#2}}

\pagestyle{empty}
\begin{document}


\section*{Naloge iz matematike}
\begin{enumerate}
    

\item Dokaži, da je enačba $ ( P \bigcap X ) \bigcup ( Q \bigcap X^c ) =  \emptyset $
rešljiva natanko tedaj, ko je $Q \subseteq P^c $.

\item Pokaži:
\begin{itemize}
    \item $M = N \iff M + N = \emptyset $
    \item $M = N = \emptyset \iff M \bigcup N = \emptyset $
\end{itemize}


\item Ali obstaja tak izjavni izraz $ A $, da bosta izraza
$ (p \bigwedge A ) \bigvee (p \Rightarrow \neg A )  $ in $ (p \Rightarrow A) \Rightarrow q $
enakovredna?

\item Dokaži:
\begin{itemize}
    \item $(A \Rightarrow B) \sim (\neg B \Rightarrow \neg A )  $
    \item $ \neg (A \bigvee B) \sim \neg A \bigwedge \neg B $
\end{itemize}

\item Poišči preneksno obliko formule $ \exists x : P(x) \bigwedge \forall x : Q(x) \Rightarrow \forall x : R(X) $.

\item Vektorja $\vec{c} = \vec{a} + 2\vec{b} $ in $\vec{d} = \vec{a} - \vec{b}$
sta pravokotna in imata dolžino $1$. Določi kot med vektorjema $\vec{a}$ in $\vec{b}$.

\item Določi definicijsko območje funkcije

    \[ f(x) = log \dfrac{x^2 + 1}{x^2 - 4x + 3} \]


\item Izračunaj
\[ \cos^2 \dfrac{3 \pi}{8} + \cos^2\dfrac{5\pi}{8} + \cos^2\dfrac{7\pi}{8} + \cos^2\dfrac{8\pi}{8} \]

\item Dokaži, da za vsa naravna števila $n$ velja
\[ \dfrac{1}{\sqrt{1}} + \dfrac{1}{\sqrt{2}} + \dots + \dfrac{1}{\sqrt{n}} \ge \sqrt{n}    \]

\item Naj bo $z$ kompleksno število, $z \neq 1$ in $\mid z \mid = 1$.
Dokaži, da je število $i \frac{z+1}{z-1}$ realno.

\item Pokaži, da je funkcija $x \mapsto \sqrt{x} $ enakomerno zvezna na $[0,\infty)$.

\item Izračunaj limito
\[ \lim_{x \to \infty}(\sin\sqrt{x+1} - \sin\sqrt{x}) \]

\item Dani sta grupi $(G,\ast)$ in $(H,\circ)$. V množici $G \times H$ definiramo operacijo
\[  (g_1,h_1) \cdot (g_2,h_2) = (g_1 \ast g_2,h_1 \circ h_2)       \]
Pokaži, da je množica $G \times H$ grupa za to operacijo.

\item Pokaži, da ima $f(x) = 3x + \sin(2x)$ inverzno funkcijo in izračunaj $(f^{-1})'(3\pi)$.

\item Izračunaj integral korenske funkcije
\[\int \dfrac{2 + \sqrt{x + 1}}{(x + 1)^2 - \sqrt{x + 1}}dx \]

\item Krivulja je podana parametrično z enačbama
\[ x(t) = \int_{1}^{t}\dfrac{\sin u}{u^2}du  \quad y(t) = \int_{1}^{t}\dfrac{\cos u}{u^2}du\] 
Izračunaj dolžino poti od točke $x = 0$ do točke, v kateri je tangenta prvič navpična.

\item Naj bo $\sum_{n=1}^{\infty} a_n$ absolutno konvergentna vrsta in $a_n \neq 1 $ za $n \in \mathbb{N}  $.
Dokaži, da sta vrsti
$$\sum_{n=1}^{\infty} \dfrac{a_n}{1 + a_n} \quad \text{in} \quad \sum_{n=1}^{\infty}\dfrac{a_{n}^{2}}{1 + a_{n}^{2}}$$
absolutno konvergentni.

\item Funkcijsko zaporedje $f_n : [a,b]\rightarrow [c,d]$ enakomerno konvergira na $[a,d]$ proti funkciji $f$.
Naj bo $g : [c,d] \Rightarrow \mathbb{R}$ zvezna. Dokaži, da funkcijsko zaporedje $g \circ f_n $
enakomerno konvergina na $[a,b]$ in določi njegovo limitno funkcijo.

\item Izračunaj limito zaporedja
\[\lim_{n \to \infty} \dfrac{\sqrt[3]{n^2 + n - 1} + \sqrt[3]{n} + n^2}{2n^2 + \sqrt{n} + 1}   \]  

\item izračunaj
$ \displaystyle \bfrac{1\quad 2\quad 3\quad 4\quad 5\quad 6}{4\quad 5\quad 2\quad 6\quad 3\quad 1}^{-2000}   $


\item Poenostavi
\[ \dfrac{\dfrac{3 + i}{2 - 2i} + \dfrac{7i}{1 - i}}{1 + \dfrac{i - 1}{4} - \dfrac{5}{2 - 3i}}          \]

\item Za dani zaporedji preveri, ali sta konvergentni
\[a_n = \underbrace{ \sqrt{2 + \sqrt{2 + \dots + \sqrt{2}}}}_{\text{n korenov}} \quad b_n = \underbrace{\sin(\sin(\dots(\sin 1)\dots))}_{\text{n sinusov}}       \]

\item Ugotovi, ali obstaja
$$\lim_{y \to 0}y \left \lgroup \dfrac{y + 1 }{y} - \sqrt{\dfrac{y^2 + 1}{y^2}} \right \rgroup $$
Pomagaj si z limitama funkcije $\dfrac{x + 1 - \sqrt{x^2 + 1}}{x}$ v $-\infty$ in $\infty$.

\item Izračunaj naslednjo determinanto $2n \times 2n$, ki ima na neoznačenih mestih ničle.
??

\item Dana je funkcija
\[ f(x,y)=  \Bigg \{ \brfrac{\frac{3x^2 - y - y^3}{x^2 + y^2};}{ a;\hfill}\quad  \brfrac{(x,y) \neq (0,0)}{(x,y)=(0,0)}          \]
\begin{itemize}
    \item Določi parameter $a$ tako, da bo $f$ zvezna.
    \item Izračunaj parcialna odvoda $f_{x}(x,y)$ in $f_{y}(x,y)$ za $(x,y) \neq (0,0)$.
    \item Izračunaj parcialna odvoda $f_{x}(0,0)$ in $f_{y}(0,0)$. Če obstaja, izračunaj limito

\[ \lim_{(x,y)\to (0,0)}\dfrac{f(x,y) - f_x (0,0) - f_y (0,0)}{\sqrt{x^2 + y^2}}            \]
    Ali je funkcija $f$ diferenciabilna?
\end{itemize}
\item Poišči vse rešitve enačbe
\begin{multline*}
     (1 + x + x^2) \cdot (1 + x + x^2 x^3 + \dots + x^9 + x^{10}) = \\ =(1 + x + x^2 + x^3 + x^4 + x^5 + x^6)^2
\end{multline*}
\item Dokaži binomsko formulo: za vsaki realni števili $a$ in $b$ in za vsako naravno število $n$ velja
%\begin{align*}
 %   \begin{split}
 %       (a+b)^n = a^n + na_{n-1}b + \dots + \bfrac{n}{k}a^{n-k}b^k + \dots + nab^{n-1} + b^n \\
%    \end{split} \\
  %  \begin{split}
  %        = \sum_{k=0}^n \bfrac{n}{k}a^{n-k}b^k 
 %   \end{split}  
%\end{align*}
\begin{equation*}
    \begin{aligned}
        a+b)^n &= a^n + na_{n-1}b + \dots + \bfrac{n}{k}a^{n-k}b^k + \dots + nab^{n-1} + b^n \\
         &= \sum_{k=0}^n \bfrac{n}{k}a^{n-k}b^k
    \end{aligned}
\end{equation*}
\item Naj bo
\begin{equation*}
    \begin{aligned}
        &G = \{z \in \mathbb{C}; z=2^k(\cos(m\pi \sqrt{2})+ i\sin(m\pi \sqrt{2})),k,m \in \mathbb{Z}\\
    &H = \{(x,y) \in \mathbb{R}^2; x,y \in \mathbb{Z}
    \end{aligned}
\end{equation*}
\begin{enumerate}
    \item Pokaži, da je $G$ podgrupa v grupi $(\mathbb{C}\setminus {0},\cdot)$neničelnih kompleksnih števil za običajno množenje.
    \item Pokaži, da je $H$ podgrupa v aditivni grupi $(\mathbb{R}^2,+)$ravninskih vektorjev za običajno seštevanje po komponentah.
    \item Pokaži, da je preslikava $f: H \to G$, podana s pravilom
    \[ (x,y) \mapsto  2^x(\cos(y\pi \sqrt{2}) + i\sin(y\pi \sqrt{2}))  \]
    izomorfizem grup ?? in ??.
\end{enumerate}

\item Nariši grafe funkcij:
 \begin{equation*}
    \begin{aligned}
        &y=x^2 - 3|x| + 2 &  &y=3\sin(\pi + x) - 2 \\
        &y=\log_2 (x - 2) + 3 & &y= 2\sqrt{x^2 + 15} + 6 \\
        &y=2^{x-3} & &y=\cos(x - 3) + sin^2(x+1)  
    \end{aligned}
 \end{equation*}             

\end{enumerate}
\end{document}
